% Straight up stealing preamble from Eli Holmes 
%%%%%%%%%%%%%%%%%%%%%%%%%%%%%%%%%%%%%%START PREAMBLE THAT IS THE SAME FOR ALL EXAMPLES
\documentclass{article}

%Required: You must have these
\usepackage{Sweave}
\usepackage{graphicx}
\usepackage{tabularx}
\usepackage{hyperref}
\usepackage{natbib}
\usepackage{pdflscape}
\usepackage{array}
\usepackage{gensymb}
%\usepackage[backend=bibtex]{biblatex}
%Strongly recommended
  %put your figures in one place
%\SweaveOpts{prefix.string=figures/, eps=FALSE} 
%you'll want these for pretty captioning
\usepackage[small]{caption}

\setkeys{Gin}{width=0.8\textwidth}  %make the figs 50 perc textwidth
\setlength{\captionmargin}{30pt}
\setlength{\abovecaptionskip}{10pt}
\setlength{\belowcaptionskip}{10pt}
% manual for caption  http://www.dd.chalmers.se/latex/Docs/PDF/caption.pdf

%Optional: I like to muck with my margins and spacing in ways that LaTeX frowns on
%Here's how to do that
 \topmargin -1.5cm        
 \oddsidemargin -0.04cm   
 \evensidemargin -0.04cm  % same as oddsidemargin but for left-hand pages
 \textwidth 16.59cm
 \textheight 21.94cm 
 %\pagestyle{empty}       % Uncomment if don't want page numbers
 \parskip 7.2pt           % sets spacing between paragraphs
 %\renewcommand{\baselinestretch}{1.5} 	% Uncomment for 1.5 spacing between lines
\parindent 0pt% sets leading space for paragraphs
\usepackage{setspace}
%\doublespacing

%%%%%%%%%%%%%%%%%%%%%%%%%%%%%%%%%%%%%%END PREAMBLE THAT IS THE SAME FOR ALL EXAMPLES

%Start of the document
\begin{document}

\bibliographystyle{/Users/aileneettinger/citations/Bibtex/styles/amnat.bst}
\title{Supplemental Materials for: How do climate change experiments actually change climate?} % Paper 1/Large group paper from Reconciling Experimental and Observational Approaches for Climate Change Impacts

\author{A.K. Ettinger, I. Chuine, B. Cook, J. Dukes, A.M. Ellison, M.R. Johnston, A.M. Panetta,\\ C. Rollinson, Y. Vitasse, E. Wolkovich}
%\date{\today}
\maketitle  %put the fancy title on
%\tableofcontents      %add a table of contents
%\clearpage
%%%%%%%%%%%%%%%%%%%%%%%%%%%%%%%%%%%%%%%%%%%%%%%%%%%
\renewcommand{\thetable}{S\arabic{table}}
\renewcommand{\thefigure}{S\arabic{figure}}

\section* {Additional methods for database development}
For our literature review, we searched Web of Science (ISI) for Topic=(warm* OR temperature*) AND Topic=(plant* AND phenolog*) AND Topic=(experiment* OR manip*). We restricted dates to the time period after the STONE database (i.e. January 2011 through March 2015). This yielded 277 new studies. We removed all passive warming studies from the list, and contacted authors for daily data. The resulting database contains daily climate data collected between 1991 and 2014 from North American and European climate change experiments (Table S1, Figure S1). 

We were unable to include the following studies because authors declined to share their data or did not respond: \citep{schwartzberg2014,moser2011,caron2015,ellebjerg2008}

\section* {Details of statistical analyses and results}
For all analyses, we use mixed-effects models implemented using the lme4 package in R, version 3.2.4 \citep{bates2015,rcoreteam2016}. Mixed-effects models, also called multi-level or hierarchical models, can account for structured data that violate the independence assumption of traditional linear regression \citep{gelman2007}. In our analyses, we use levels/groupings of experimental site, year, and day of year (doy) to account for this mutual dependence among data points. To test for significance of fixed effects in our models, we use Type II tests for models including only main effects and Type III tests for models including interactions, as well as main effects. 
\subsection* {Analysis of effects of time and space on local experimental climate}
To test how treatment effects vary spatially (i.e., among blocks within a study) and temporally (i.e., among years within a study), we used data from the four studies in the C3E database that used blocked designs. We fit linear mixed-effect models with mean daily soil temperature, minimum daily air temperature, and maximum daily air temperature as response predictors (Figure 3 in the main text). For temporal models, we included fixed effects of temperature treatment, year, and their interaction; random effects were site and block nested within site (intercept-only structure, Table \ref{table:blocks_time}). For spatial models, we included fixed effects of temperature treatment, block, and their interaction; random effects were site and year nested within site (intercept-only structure, Table \ref{table:blocks_space}). Both of these models excluded data from plots with precipitation treatments. 
\subsection* {Analysis of effects of infrastructure on local experimental climate}
To test how infrastructure affects local climate, we compared temperature and soil moisture data from the studies in the C3E database that monitored climate in two types of control plots: structural controls (i.e., `shams' or `disturbance controls,'
which contained all the warming infrastructure, such as soil cables or infrared heating units but with no heat
applied) and ambient controls with no infrastructure added. These five studies occurred at two sites: Duke Forest and Harvard Forest \citep{farnsworth1995,clark2014a,marchin2015,pelini2011}. We fit linear mixed effects models by month with mean daily soil temperature, minimum and maximum daily air and soil temperature (\citet{farnsworth1995} did not measure these predictors so there are only four different studies in these models), and soil moisture as response predictors. The fixed explanatory predictor was control type (sham or ambient). To allow for both mean differences in temperature and the effect of control to to vary among sites and years, random effects were site and year nested within site, modeled with a random slopes and random intercepts structure. 
We found that experimental structures altered above-ground and soil temperatures in opposing ways: above-ground temperatures were higher in the structural controls, compared with ambient conditions with no structures installed, whereas soil temperatures were lower in the structural controls compared with ambient soil (Figure 4 in the main text).  In addition, soil moisture was lower in structural controls compared with ambient conditions. These general patterns were consistent across the different temperature models we fit (mean,
minimum, and maximum soil and air temperatures), although the magnitude varied across months, as well as among studies. We show summaries from models fit to the entire year (Tables \ref{table:shamamb_soiltemp}, \ref{table:shamamb_airtemp}, \ref{table:shamamb_soilmois}), as well as summaries from models fit to each month of data, as is shown in Figure 4 in the main text (Tables \ref{table:shamamb_stempm}, \ref{table:shamamb_atempm}, \ref{table:shamamb_soilmoism}).

\subsection* {Analysis of effects of precipitation treatments on above-ground temperature}
Of the twelve experiments in the C3E database, four manipulated precipitation and measured above-ground temperature. To examine the effects of precipitation treatment on above-ground temperature, we fit linear mixed effect models to data from these four sites with above-ground temperature (daily minimum and maximum) as the response variables. Predictors were precipitation treatment (a continuous fixed effect, which ranged from 50 to 200 \% of ambient for these four studies), target warming (a continuous fixed effect, which ranged from 0 to 4 \degree C for these four studies), and their interaction. To account for methodological and other differences among site, we included site as a random effect, with year and doy nested within site to account for the non-independent nature of measurements taken on the same day within sites. We used a random intercept model structure, (Table \ref{table:preciptemp}). 

\subsection* {Analysis of effects of experimental warming on soil moisture}
Of the twelve experiments in the C3E database, ten measured and reported soil moisture. To examine the effects of target warming treatment on soil moisture, we fit linear mixed effects models to data from these ten sites, excluding plots with precipitation treatments. We first fit a model with soil moisture as the response and a predictor of target warming (this was a continuous fixed effect, which ranged from 0 to 5.2 \degree C for these 10 studies). To account for methodological and other differences among site, we included site as a random effect, with year and doy nested within site to account for the non-independent nature of measurements taken on the same day within sites.  We used a random slope and intercept model structure, to allow the effect of target warming to vary among sites (Table \ref{table:targsoilmois}). 

\par In addition to testing how experimental warming influenced soil moisture, we also tested how experimental structures influenced soil moisture. We compared the soil moisture measured in structural controls to both ambient controls and warmed plots by fitting a model with categorical fixed effects of ``ambient," ``structural control," and ``warmed."  We again included site as a random effect, with doy nested within site to account for the non-independent nature of measurements taken on the same day within sites, and used a random intercept structure (Table \ref{table:warmsoilmois}). 

\subsection* {Analysis of budburst phenology}
We wanted to investigate how using measured plot-level climate variables, as opposed to target warming, alters estimates of temperature sensitivity in ecology. To do this, we fit two different types of models to data from the five study sites in the C3E database that recorded above-ground temperature and soil moisture, as well as phenology data (doy of budburst). We focus on budburst, as this phenological phase was reported most commonly among studies in the C3E database. For all models, we accounted for non-independence by including species, site, and year nested within site as intercept-only random effects (Table \ref{table:bbmods}). The target warming model included only one explanatory variables (the target amount of warming).  We compared this to models with mean annual measured above-ground temperature (offset by subtracting the minimum temperature across all studies and plots, to make model intercepts more similar), mean winter (January-March) soil moisture, and their interaction as explanatory variables. The slope for temperature in the measured climate model can be directly compared to the slope for target warming in the target warming model because the units are the same (change in budburst, in days/\degree C).
\par To determine which specific above-ground temperature variable to include, we compared AICs of models for with four different temperature variables (mean annual minimum and maximum temperatures, mean January-March minimum and maximum temperatures). The model with mean annual minimum temperature, mean January-March soil moisture, and their interaction provided the best model fit (lowest AIC, highest explained variation, Table \ref{table:bbaic}), so we discuss and interpret that model in the main text, summarize it in Table \ref{table:bbmods}, and present its coefficients in Figure 7. 



\bibliography{/Users/aileneettinger/citations/Bibtex/mylibrary}

\section* {Supplemental Tables} 

\begin{table}[b]
  \caption{\textbf{Sites included in the C3E database}. Experimental sites correspond to the map (Figure 1, main text). We give the study ID, location, source, years of data included, and warming type used in the study. Note that some sites may have multiple sources; however, we list only one here. Note that we were unable to include the following studies because authors declined to share their data or did not respond: \citep{schwartzberg2014,moser2011,caron2015,ellebjerg2008}.}
\begin{footnotesize} 
   \begin{tabular}{| p{1.2cm} | p{5.7cm} | p{3.5cm} | p{1.5cm} | p{2cm} |}
    \hline
  study id & location & source & data years & warming type \\ \hline
    exp01 & Waltham, MA, USA & \cite{hoeppner2012} & 2009-2011 & infrared\\ \hline
    exp02 & Montpelier, France & \cite{morin2010} & 2004 & infrared\\ \hline
    exp03 & Duke Forest, NC, USA & \cite{clark2014a} & 2009-2014 & forced air and soil warming\\ \hline
    exp04 & Harvard Forest, MA, USA & \cite{clark2014a} & 2009-2012 & forced air and soil warming\\ \hline
    exp05 & Jasper Ridge Biological Preserve, CA, USA & \cite{cleland2007} & 1998-2002 & infrared\\ \hline
    exp06 & Rocky Mountain Biological Lab, CO, USA & \cite{dunne2003} & 1995-1998 & infrared\\ \hline
    exp07 & Harvard Forest, MA, USA & \cite{pelini2011} & 2010-2015 & forced air \\ \hline
    exp08 & Harvard Forest, MA, USA & \cite{farnsworth1995} & 1993 & soil warming \\ \hline
    exp09 & Stone Valley Forest, PA, USA & \cite{rollinson2012} & 2009-2010 & infrared \\ \hline
    exp10 & Duke Forest, NC, USA & \cite{marchin2015} & 2010-2013 & forced air \\ \hline
    exp11 & Rocky Mountain Biological Lab, CO, USA & \cite{price1998} & 1991-1994 & infrared\\ \hline
    exp12 & Kessler Farm Field Laboratory, OK, USA & \cite{sherry2007} & 2003 & infrared\\ \hline
     \end{tabular} 
    \end{footnotesize}  
    \end{table}
  
\clearpage

\newcolumntype{L}[1]{>{\raggedright\let\newline\\
\arraybackslash\hspace{0pt}}m{#1}}
\newcolumntype{C}[1]{>{\centering\let\newline\\
\arraybackslash\hspace{0pt}}m{#1}}
\newcolumntype{R}[1]{>{\raggedleft\let\newline\\
\arraybackslash\hspace{0pt}}m{#1}}
\newcolumntype{P}[1]{>{\raggedright\tabularxbackslash}p{#1}}

\begin{footnotesize} 
% latex table generated in R 3.4.2 by xtable 1.8-2 package
% Fri Feb  9 12:57:48 2018
\begin{table}[ht]
\centering
\caption{\textbf{Experimental sites included in the C3E database}. Experimental sites correspond to the map (Figure 1, main text). We give the study ID, location, source, years of data included, warming type,target warming treatment (\degree C), precipitation treatment (percent of ambient), method of above-ground temperature measurement (with height of measurement, in cm, for air), depth of soil temperature measurement (cm), and depth of soil moisture measurement (cm) used in each study. Note that some sites may have multiple sources; however, we list only one here.} 
\label{tab:methods}
\begingroup\footnotesize
\begin{tabular}{|p{0.04\textwidth}|p{0.17\textwidth}|p{0.08\textwidth}|p{0.10\textwidth}|p{0.07\textwidth}|p{0.07\textwidth}|p{0.07\textwidth}|p{0.07\textwidth}|p{0.07\textwidth}|}
  \hline
study & location & data years & warming type & warming treatment & precip treatment & above-ground temperature & soil temperature depth & soil moisture depth \\ 
  \hline
exp01 & Waltham, MA, USA & 2009-2011 & infrared & 1, 2.7, 4 & 50, 100, 150 & canopy & 2, 10 & 30 \\ 
  exp02 & Montpelier, France & 2004 & infrared & 1.5, 3 & 70, 100 &   &   & 15, 30 \\ 
  exp03 & Duke Forest, NC, USA & 2009-2014 & forced air and soil warming & 3, 5 &   & air (30) & 10 & 30 \\ 
  exp04 & Harvard Forest, MA, USA & 2009-2012 & forced air and soil warming & 3, 5 &   & air (30) & 10 & 30 \\ 
  exp05 & Jasper Ridge Biological Preserve, CA, USA & 1998-2002 & infrared & 1.5 & 100, 150 &   & 15 & 15 \\ 
  exp06 & Rocky Mountain Biological Lab, CO, USA & 1995-1998 & infrared & 1.5 &   &   & 12, 25 & 12, 25 \\ 
  exp07 & Harvard Forest, MA, USA & 2010-2015 & forced air & 1.5-5.5 &   & air (22) & 2, 6 & 30 \\ 
  exp08 & Harvard Forest, MA, USA & 1993 & soil warming & 5 &   &   & 5 &   \\ 
  exp09 & Stone Valley Forest, PA, USA & 2009-2010 & infrared & 2 & 100, 120 & surface & 3 & 8 \\ 
  exp10 & Duke Forest, NC, USA & 2010-2013 & forced air & 1.5-5.5 &   & air (22) & 2, 6 & 30 \\ 
  exp11 & Rocky Mountain Biological Lab, CO, USA & 1991-1994 & infrared & 1 &   &   & 12 &   \\ 
  exp12 & Kessler Farm Field Laboratory, OK, USA & 2003 & infrared & 4 & 100, 200 & air (14) & 7.5, 22.5 & 15 \\ 
   \hline
\end{tabular}
\endgroup
\end{table}\end{footnotesize} 
\clearpage

\newcolumntype{L}[1]{>{\raggedright\let\newline\\
\arraybackslash\hspace{0pt}}m{#1}}
\newcolumntype{C}[1]{>{\centering\let\newline\\
\arraybackslash\hspace{0pt}}m{#1}}
\newcolumntype{R}[1]{>{\raggedleft\let\newline\\
\arraybackslash\hspace{0pt}}m{#1}}
\newcolumntype{P}[1]{>{\raggedright\tabularxbackslash}p{#1}}


%%%%%%%%%%%%%%%%%%%%%%%%%%%%%%%%%%%%%%%%
\end{document}
%%%%%%%%%%%%%%%%%%%%%%%%%%%%%%%%%%%%%%%%
