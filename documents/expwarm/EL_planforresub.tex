\documentclass[12pt,a4paper]{article}
\usepackage[top=1.00in, bottom=1.0in, left=1in, right=1in]{geometry}
\usepackage{amsmath}

\begin{document} 

\noindent {\bf Plan for resubmission of Experimental Climate Change manuscript to \\ \emph{Ecology Letters} }\\
\noindent {\underline {Changes to make}:}\\
\begin{enumerate}
\item Soften text about climate change experiments/intentions.(Rev 1\&3) 
\begin{enumerate}
\item alternative to ``unintended" (Rev. 1)
\item Ed: ``mischaracterizes the intentions of experimentalists and is unjustifiably antagonistic."
\item Make clear that not all experiments had an explicit ``Target" temperature (Rev. 1)
\item Add more about analyses used to date? (ANCOVA, multiple regression) (Rev. 1).
\item make point that studies DO tend to do detailed analyses of microclimate (Rich et al 2015, Harte et al 1995), just in separate papers rather than jointly with the biological data. 
\item `` do believe that individual global change experimental papers typically discuss artifacts and actual vs target amounts of temperature, precipitation (or CO2 or N or S deposition) change in global change experiments" (Rev 2)
\item Add text to ``acknowledge diverse, alternative goals (does one want to isolate a single treatment variable or have a more ecologically realistic treatment that also
simulates indirect effects? See below for more about this), rather than assuming that experiments that include indirect effects have major problems." (Rev 3)
\item Add examples of when  experimental treatments ``do result in similar responses as natural variation in the same driver," (Rev 3)- when are experiments good (see references)
\item  frame concepts about what we should expect across different temporal and spatial scales when comparing experiment to observed patterns- i.e. we should
not always expect them to be the same, as long-term indirect effects are part of the long-term observational record but not of the shorter-term experiments... In other words, this may be more of an issue of how people interpret experiments versus observations, rather than any intrinsic problem with either." (Rev 3)
\end{enumerate}
\item Better separation/description/interpretation of different warming methods (Rev 2\&3, Ed)
\begin{enumerate}
\item Distinguish between above and below-ground heating in figures- i.e. in figs 1 and 4, describe in legend; in figs 2 and 3:  use different shapes for different heating types (Can add color if necessary)- be consistent with these shapes in both figs. 
\item More discussion of differences between two methods in text, where appropriate, and ``more precision and accuracy in the assessment of their strengths and weaknesses." (Ed) 
\item ``clearly identifying (and separately interpreting) which studies directly and independently heated both above and belowground, just aboveground, or just belowground; and which studies attempted to maintain a set elevation of temperature versus applied constant inputs of energy (e.g. constant wattage for infrared heaters)." (Rev 3)
\item Re-do analyses of soil moisture, infrastructure effects with "warming type" and its interaction added. If differences found by warming type, include a discussion of this in each section (Rev 3)
\item add table of different warming types and range of warming experienced, and studies included in analysis (Rev 3)
\end{enumerate}
\item Add discussion about the limited number of studies in our dataset (Rev 2\&3, Ed) .
\begin{enumerate}
\item more case studies (Rev 2): would have been great to have but additional researchers were not willing to share their data. we used explicit search criteria in web of science. 
\item it is our hope that the paper will spur additional pooling of data and analyses across an even larger dataset.  hopefully more studies will come out of the woodwork as a result!
(``Perhaps the authors can comment that these types of artifact issues should receive more attention in future, more robust syntheses or meta-analyses." (Rev 2))
\end{enumerate}
\item Rewrite soil moisture/projected climate change discussion to reflect uncertainty about soil moisture in east (Rev 3,Ed)
\begin{enumerate}
\item ask ben about Rev 3's interpretation of Seager et al. and get his suggestions for rewriting
\end{enumerate}
\item Add citations 
\begin{enumerate}
\item Harte et al Ecol Appli. (Rev. 1)- can cite this in several places (shallower soils drier, this study does not use sham controls, as an example of detailed plot-level analyses to understand variation in experimental warming effects, in support of the idea that it is difficult to completely separate temperature and precip/moisture treatments- higher soil moisture reduces temperatures, vegetation effect on microclimate, also they state that ``techniques for extrapolating local results to regional and global scales are needed") . this is a great study/paper/detailed analysis- but just at one site! meta-analysis still useful! also, this does not measure air/canopy/surface temp- only soil). 
\item Rich et al (Rev. 3)- can cite as an example of a study that found no need for structural controls because there was no difference ``thermally or biologicall" (not clear how they assessed no difference, however- at what temporal/spatial scale? - and data not shown)
\item Aspinwall et al, Reich et al 2016 (Rev 3): examples of when experimental response (acclimation to warming) appears to be the same as observational response across seasonal changes
\item Carey at al 2016 (experimental warming reduces soil moisture)
\end{enumerate}

\item Edit Supplement
\begin{enumerate}
%\item Remove authors who did not provide data from supplement (Rev 2)%Lizzie says keep this in
\item Add ecosystem to Table S1 (Rev 3)
\item Combine Tables S1 and S2 (Rev 2)
\item Table S2: add to legend something about targets being defined by researchers (Rev 2)
\end{enumerate}
\item Other small/detailed edits 
\begin{enumerate}
\item Clarify whether soil and air, or just soil respond to precipitation (I think both!) ``Lines 198-202 ignore the bigger issue here which is the soil, not air, temperature response triggered by precipitation.  Lines 215-216 will mislead the reader into thinking that carbon feedback is a local effect, and it ignores the evidence that both negative and positive feedbacks can arise from the processes described."(Rev. 1)
\item add discussion about inherent variability in soil moisture in main text (Rev 2)
\item Figure 5 ``Figure 5 is presented as a
problem of indirect effects. If the indirect effects are ecologically realistic ones, then one might
argue that this is THE VALUE of realistic long-term field experiments, to be able to assess in an
ecologically realistic way both the direct effects and the indirect effects that cascade through
ecosystems over time." (Rev. 3)
\item Fig 6, clarify which studies went into this statistical model. add discussion of value of separating direct from indirect effects moisture effects, and perhaps add nuance to interpretation? (see Rev 3 comments). 
\item see Rev 3's extensive comments
\item Change title of database/paper (Rev 2: should be ``microclimate" not  ``climate")
\item Try to add SPRUCE data (contact authors)

\end{enumerate}
\item  {\underline {Things to disagree with/respond to in point-by-point only}:}\\
\begin{enumerate}
\item ``Lines 243-245 repeat the discredited notion that all warming experiments generate phenological responses that differ from observational responses at the same site." (Rev. 1)
\item ``More detail on where measurements are made:  We know that the soil warming experiments will best achieve their target temperatures at the depth to which they are controlled (typically 5-10 cm depth), and that the warming will decline with depth.  I expect the same is true for air temperature. It is not clear from the figures at what depth soil (or air) temperatures were measured." Point Rev 2 to supplement (make sure that it is cited in main text).
\item `` Can authors use ``air" temperature instead of ``above ground temperature? " (Rev 2): make sure text and/or legend(s) explains what above ground includes (canopy, air, surface)
 \end{enumerate}
%\noindent {\underline {Notes from meeting with Lizzie}:}\\

%\item do not touch the "discredited" paper- Lizzie's take is that john chase wants me to addres the tone issues of reviewer but nothing else
%\item go back to papers and see how often people talk about artifacts of temperatures, characterized. do the papers mention the artifacts described, wattage versus target temperatures wattage per m2, stats, lizzie could help with this. 
%\item compare air/soil cables to infrared. make clear that this is not a full comparison because air/soil cables are one NSF grant
%\item move methods to front
%\item add in some passive OTC data somehow? ask sarah if daily database of OTC data. 
%\item people ask us why certain studies are not included. this is why it is included. do not remove
%\item have table for study design, habitats/ecosystem, reported levels of warming achieved, anything else that's comparable? have a box for this. use this as an opportunity to restate methods. 
%\item harder to collect climate data in OTCs- they exist because they are cheap and employed in locations that are difficult to get to. snow melt. OTC has been very upfront about indirect effects. artifacts are so big that harder to ignore. 
%\item talk to ben about how to handle get more citations
%\item what was warmed: air vs. canopy
%\item keep table S1/s2 in supplement.
%\item in new table of warming types, add experiments included here. 

 
\end{enumerate}
\end{document}
