\documentclass[11pt,a4paper]{letter}
\usepackage[top=1.00in, bottom=1.0in, left=1in, right=1in]{geometry}
\usepackage{graphicx}
\usepackage{natbib}

\address{1300 Centre Street \\ Boston, MA, 20131}

\begin{document}
\bibliographystyle{/Users/aileneettinger/citations/Bibtex/styles/nature.bst}

\begin{letter}{}
\includegraphics[width=0.5\textwidth]{/Users/aileneettinger/Dropbox/Documents/Work/AA_heading.pdf}
\pagenumbering{gobble}

\opening{Dear Dr. Chase, Dr. Hillebrand, and members of the editorial board:}
Please consider our paper, entitled ``How do climate change experiments actually change climate?'' for publication as a Review \& Synthesis in \emph{Ecology Letters}. Our proposal to submit this manuscript was accepted by Dr. Chase on October 5, 2017. The paper is coauthored by I. Chuine, B.I. Cook, J.S. Dukes, A.M. Ellison, M.R. Johnston, A.M. Panetta, C.R. Rollinson, Y. Vitasse, E.M. Wolkovich, and myself.

The biological impacts of climate change have been widely observed around the world, from shifting species' distributions to altered timing of important life events, and remain a major area of ecological research. With growing evidence and interest in these impacts, ecologists today are challenged to make quantitative, robust predictions of the ecological effects of climate change. One of the most important methods to achieve this goal is field-based climate change experiments that alter temperature and precipitation. The utility of these experiments, however, is directly dependent on the climate change they produce and how accurately researchers assess and present these changes. 

For over three decades, ecologists have relied on field climate change experiments to understand and forecast ecological impacts of climate change. These experiments are still a prevalent current method, used across diverse sub-disciplines from ecophysiology (1) % reich2015 
to foodweb ecology (2), %\citep{barton2009},  
for cutting-edge climate change research. They critically offer the ability to create ``no-analog'' climate scenarios forecasted for the future, to isolate effects of temperature and precipitation from other environmental changes, and to examine non-linear responses to climatic changes. Yet, increasingly these experiments have been shown to estimate effects much smaller than those seen in long-term observational studies (3). %\citep{wolkovich2012}. 
Despite calls for improved methods (4,5), %{beier2012,kreyling2014}, 
even sophisticated approaches appear to suffer from this discrepancy (6). %\citep{menke2014}. 
Such results highlight the need to synthesize across studies to assess how realistically experiments can alter climate conditions, as well as develop novel approaches for applying experimental results to forecasting biological impacts of global climate change. 

Here we describe how experimental climate change results may be interpreted in misleading ways, especially through the common practice of summarizing and analyzing only the mean changes across treatments.  We show that such methods mask important variation in treatment effects over space and time, using a new database of high-resolution climate data from some of the most advanced field-based climate change experiments. Our database assembles daily climate data from 12 active warming experiments, containing an estimated 44 study years and 11594 study days of air and soil temperature and soil moisture data). 

We also find that secondary, unintended treatment effects, which are rarely described or interpreted (e.g. soil drying with warming treatments), may under- or over-estimate climate change impacts. The implications of these complexities are likely to have important biological consequences, across diverse ecological systems. We describe a case study of spring plant phenology, in which a simple mean-focused analysis, ignoring secondary effects, leads to inaccurate quantification of species' sensitivities to changes in temperature. We present our recommendations for future experimental design, analytical approaches, and data sharing that we believe will improve the ability of climate change experiments to accurately identify and forecast species' responses.

Our author team brings together an international and interdisciplinary team of researchers, which bridges perspectives from ecology, climatology, and land surface modeling. It is comprised of many of the scientists who execute major warming experiments, as well as those who have raised concerns over the findings of such experiments.  We expect our proposed Review \& Synthesis will lead to improved mechanistic understanding of climatic drivers of biological responses, and inspire innovative experimental design and analysis; we hope you will consider it for \emph{Ecology Letters}.

Sincerely,\\

\includegraphics[scale=1]{/Users/aileneettinger/Dropbox/Documents/Work/AileneEttingerSignature.png} \\
Ailene Ettinger
Postdoctoral Fellow, Arnold Arboretum of Harvard University \& Biology Department, Tufts University

\noindent \emph{References mentioned in cover letter}
\begin{footnotesize}
\begin{enumerate}
\item Reich,  P. B.,  K. M. Sendall,  K. Rice,  R. L. Rich,  A. Stefanski,  S. E. Hobbie,and R. A. Montgomery. 2015.  Geographic range predicts photosynthetic and growth response to warming in co-occurring tree species.  \emph{Nature Climate Change} 5:148-152.
\item Barton,  B.  T.,  and  O.  J.  Schmitz.  2009.   Experimental  warming  transforms multiple predator effects in a grassland food web.   \emph{Ecology Letters} 12:1317-1325.
\item Wolkovich,  E.  M.,  B.  I.  Cook,  J.  M.  Allen,  T.  M.  Crimmins,  J.  L.  Betan-court, S. E. Travers, S. Pau, J. Regetz, T. J. Davies, N. J. B. Kraft, T. R.Ault, K. Bolmgren, S. J. Mazer, G. J. McCabe, B. J. McGill, C. Parmesan,N. Salamin, M. D. Schwartz, and E. E. Cleland. 2012.  Warming experiments underpredict plant phenological responses to climate change.  \emph{Nature} 485:494-497.
\item Beier,   C.,   C.   Beierkuhnlein,   T.   Wohlgemuth,   J.   Penuelas,   B.   Emmett,C. Korner, H. Boeck, J. H. Christensen, S. Leuzinger, I. A. Janssens, et al. 2012.  Precipitation manipulation experiments: challenges and recommendations for the future.  \emph{Ecology Letters} 15:899-911.
\item Kreyling, J., A. Jentsch, and C. Beier. 2014.  Beyond realism in climate change experiments: gradient  approaches  identify  thresholds  and  tipping  points. \emph{Ecology Letters} 17:125.
\item Menke, S. B., J. Harte, and R. R. Dunn. 2014. Changes in ant community com-position caused by 20 years of experimental warming vs. 13 years of natural climate shift.  \emph{Ecosphere} 5:1-17.

\end{enumerate}
\end{footnotesize}



\end{letter}
\end{document}
