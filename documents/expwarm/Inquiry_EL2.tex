\documentclass[12pt,a4paper]{letter}
\usepackage[top=1.00in, bottom=1.0in, left=1in, right=1in]{geometry}
\usepackage{graphicx}
\usepackage{natbib}

\address{1300 Centre Street \\ Boston, MA, 20131}

\begin{document}
\begin{letter}{}

%\signature{}

\bibliographystyle{/Users/aileneettinger/citations/Bibtex/styles/nature}
\renewcommand{\refname}{\CHead{}}

\includegraphics[width=0.5\textwidth]{//Users/aileneettinger/Dropbox/Documents/Work/AA_heading.pdf}
\pagenumbering{gobble}

\opening{Dear Dr. Drake:}
We propose an Ideas \& Perspectives piece for \emph{Ecology Letters} on experimental climate change. 

The biological impacts of climate change have been widely observed around the world \citep{ipcc2013}, from shifting species' distributions to altered timing of important life events, and remain a hot area of ecological research \citep{goring2017}. With growing evidence and interest in these impacts, ecologists today are challenged to make quantitative, robust predictions of the ecological effects of climate change. One of the most important methods to achieve this goal is field climate change experiments \citep{harte1995,cleland2006,Hoeppner2012} .

For over three decades, ecologists have relied on field climate change experiments, which alter temperature via active warming methods such as infra-red heaters, to understand and forecast ecological impacts of climate change. These experiments are still a prevalent current method, used across diverse subdisciplines from ecophysiology\citep{reich2015} to foodweb ecology \citep{barton2009}, for cutting-edge climate change research. They critically offer the ability to create ``no-analog'' climate scenarios forecasted for the future, to isolate effects of temperature and precipitation from other environmental changes, and to examine non-linear responses to climatic changes. Yet, increasingly these experiments have been shown to estimate effects much smaller than those seen in long-term observational studies\citep{wolkovich12}. Such results highlight the need for new ideas and methods to rigorously assess how these experiments alter climate conditions, as well as novel approaches for applying the findings from these assessments to forecasting biological impacts of global climate change. 

We propose an Ideas \& Perspectives piece that  brings together an international and interdisciplinary team of researchers to present the nuance of how climate change experiments actually alters soil and air climate and what this means for our ability to extrapolate from such experiments. Our author team bridges perspectives from ecology, climatology, and land surface modeling and is comprised of many of the scientists who execute major warming experiments, as well as those who have raised concerns over the findings of such experiments. We review how results from these experiments are frequently interpreted in misleading ways, in part because the common practice of summarizing and analyzing only the mean changes across treatments hides variation in treatment effects over space and time. In addition, we highlight how secondary, unintended treatment effects that are rarely described or interpreted (e.g. soil drying with warming treatments) may under- or over-estimate climate change impacts. All of these complications challenge our interpretation of how experimental warming studies can be applied to forecast effects of climate change. To support this, we would present the first meta-analysis of high-resolution climate data from field-based climate change experiments. We have assembled a new database of daily climate data from 12 active warming experiments, containing 44 study years and 11594 study days of air and soil temperature and soil moisture data. 

We believe there is a need to rethink the design and interpretation of climate change experiments. In our proposed paper, we would make specific recommendations for future experimental design, analysis, and data sharing that will improve the ability of climate change experiments to accurately identify and forecast species' responses to changes in climate. We expect our proposed Ideas \& Perspectives piece will lead to improved mechanistic understanding of climatic drivers of biological responses, and inspire innovative experimental design and analysis. 

Sincerely,\\

\includegraphics[scale=1]{/Users/aileneettinger/Dropbox/Documents/Work/AileneEttingerSignature.png} \\
Ailene Ettinger
Postdoctoral Fellow, Arnold Arboretum of Harvard University \& Biology Department, Tufts University

\newpage
\noindent {\bf How do climate change experiments actually change climate?}\\
\\
\noindent \emph{Authors:} A.K. Ettinger, I. Chuine, B.I. Cook, J.S. Dukes, A.M. Ellison, M.R. Johnston, A.M. Panetta, C.R. Rollinson, Y. Vitasse, \& E.M. Wolkovich
\\
% EMW: Made a couple adjustments below to stress I&P aspect of paper, but you should make a few more I think.
%I&Ps are "novel essays for a general audience"
%Proposals should be no more than 300 words long, describe the nature and novelty of the work, the contribution of the proposed article to the discipline, and the qualifications of the author(s) who will write the manuscript. Proposals should be sent to the Editorial Office (ecolets@cefe.cnrs.fr). 

%Ecology Letters is particularly interested in novel essays expressing new ideas and perspectives that will appeal to a wide ecological audience. It is important that Ideas and Perspectives be focused on a topic of current interest. We are interested in new ideas, emerging frameworks, and controversial perspectives on hot areas of research. There is a need to present a view that is sufficiently complete to convince reviewers of the value of the contribution. There is the same expectation for the novelty of Ideas and Perspectives as for Letters. Those articles principally reviewing a topic, those that are just as statement of opinion, and those primarily discussing the author's own work will not be considered. Articles that are successful usually present a quantitative analysis, as a way of introducing a new perspective in ecology. Authors interested in submitting such a manuscript should first send a one-paragraph proposal (no more than 300 words) to the Editorial Office (see above). 
The biological impacts of climate change have been widely observed around the world, from shifting species' distributions to altered timing of important life events, and remain a hot area of ecological research. With growing evidence and interest in these impacts, ecologists today are challenged to make quantitative, robust predictions of the ecological effects of climate change. One of the most important methods to achieve this goal is field-based climate change experiments that alter temperature and precipitation (e.g., with infrared heaters, rain shields, and supplemental watering). The utility of these experiments is directly dependent on the climate change they produce; yet, a rigorous assessment of how these experiments alter climate conditions has never been conducted. We describe how experimental results may be interpreted in misleading ways. Using a new database of daily climate data from 12 active warming experiments, we show that the common practice of summarizing and analyzing only the mean changes across treatments can hide important variation in treatment effects over space and time. Furthermore, treatments produce unintended secondary effects, such as soil drying in conjunction with warming. The implications of these complexities are rarely explored, but likely to have important biological consequences. We describe a case study of spring plant phenology, in which such secondary effects lead to inaccurate quantification of species' sensitivities to changes in temperature. We present our recommendations for future experimental design, new analytical approaches, and data sharing that we believe will improve the ability of climate change experiments to accurately identify and forecast species' responses.


\noindent \emph{References mentioned in cover letter}

\begin{footnotesize}
{\def\section*#1{}
\bibliography{/Users/aileneettinger/citations/Bibtex/mylibrary}
}
\end{footnotesize}

\end{letter}
\end{document}
