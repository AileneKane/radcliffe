\documentclass[12pt,a4paper]{letter}
\usepackage[top=1.00in, bottom=1.0in, left=1in, right=1in]{geometry}
\usepackage{graphicx}

\signature{Elizabeth M Wolkovich}
\address{1300 Centre Street \\ Boston, MA, 20131}

\begin{document}
\begin{letter}{}
\includegraphics[width=0.5\textwidth]{//Users/aileneettinger/Dropbox/Documents/Work/AA_heading.pdf}
\pagenumbering{gobble}

\opening{Dear Dr. Drake:}
We propose an Ideas \& Perspectives piece for \emph{Ecology Letters} on experimental climate change . For over three decades, ecologists have relied on field climate change experiments, which alter temperature via active warming methods such as infra-red heaters, to understand and forecast ecological impacts of climate change.  These experiments critically offer the ability to create ''no-analog" climate scenarios forecasted for the future, to isolate effects of temperature and precipitation from other environmental changes, and to examine non-linear responses to climatic changes. The utility of these experiments is directly dependent on the climate change they produce, yet a rigorous assessment of how these experiments alter climate conditions has never been conducted. 

We propose a meta-analysis of high-resolution climate data from field-based climate change experiments. We have assembled a new database of daily climate data from 12 active warming experiments, containing XX study years and XX study days of air and soil temperature and soil moisture data. We believe that results from these experiments are frequently interpreted in misleading ways, because the common practice of summarizing and analyzing only the mean changes across treatments hides variation in treatment effects over space and time. In addition, we have identified secondary, unintended treatment effects that are rarely described or interpreted (e.g. soil drying with warming treatments). All of these complications challenge our interpretation of how experimental warming studies can be applied to forecast effects of climate change.

We believe there is a need to rethink the design and interpretation of climate change experiments. In our proposed paper, we would make specific recommendations for future experimental design, analysis, and data sharing that will improve the ability of climate change experiments to accurately identify and forecast species' responses to changes in climate. We expect that future analyses of this database will lead to improved mechanistic understanding of climatic drivers of biological responses, and inspire innovative experimental design and analysis. This paper brings together an international and interdisciplinary team of researchers that bridges perspectives from ecology, climatology, and land surface modeling. Importantly it is comprised of many of the scientists who executed major warming experiments, as well as those who have raised concerns over the findings of such experiments.

Sincerely,\\

\includegraphics[scale=1]{/Users/aileneettinger/Dropbox/Documents/Work/AileneEttingerSignature.png} \\
Ailene Ettinger
Postdoctoral Fellow, Arnold Arboretum of Harvard University \& Biology Department, Tufts University

\newpage
\noindent {\bf How do climate change experiments actually change climate?}\\
\\
\noindent \emph{Authors:} A.K. Ettinger, I. Chuine, B.I. Cook, J.S. Dukes, A.M. Ellison, M.R. Johnston, A.M. Panetta, C.R. Rollinson, Y. Vitasse, \& E.M. Wolkovich
\\
The biological impacts of climate change have been widely observed around the world, from shifting species' distributions to altered timing of important life events. With growing evidence and interest in these impacts, ecologists today are challenged to make quantitative, robust predictions of the ecological effects of climate change. One of the most important methods to achieve this goal is field-based climate change experiments that alter temperature and precipitation (e.g., with infrared heaters, rain shields, and supplemental watering). The utility of these experiments, however, is directly dependent on the climate change they produce. We describe how these experimental results may be interpreted in misleading ways. Using a new database of daily climate data from 12 active warming experiments, we find that the common practice of summarizing and analyzing only the mean changes across treatments hides important variation in treatment effects over space and time. Furthermore, treatments produce unintended secondary effects, such as soil drying in conjunction with warming. The implications of these complexities are rarely explored, but likely to have important biological consequences. We describe a case study of spring plant phenology, in which such secondary effects lead to in accurate quantification of species' sensitivities to changes in temperature. Based on our findings, we present several recommendations for future experimental design, analysis, and data sharing that we believe will improve the ability of climate change experiments to accurately identify and forecast species' responses.
\end{letter}
\end{document}
