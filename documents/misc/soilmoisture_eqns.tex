\documentclass[12pt,a4paper]{article}
\usepackage[top=1.00in, bottom=1.0in, left=1in, right=1in]{geometry}
\usepackage{amsmath}

\setlength\parindent{0pt}

\begin{document}

\noindent I think this is what I was thinking; I show an example assuming a base temp of 20C and base soil moisture of 0.10 and a warming treatment of 5C that dries soil moisture by 0.3:

With no warming treatment applied:
\begin{equation}
y_{Z}=\alpha + \beta_1(20)+\beta_2(0.10)
\end{equation}

With the warming treatment applied, but soil moisture not affected by warming (aka, what is generally assumed):
\begin{equation}
y_{Y}=\alpha + \beta_1(25)+\beta_2(0.10)
\end{equation}

With the warming treatment applied, but with soil moisture effect included:
\begin{equation}
y_{W}=\alpha + \beta_1(25)+\beta_2(0.07)
\end{equation}

Then you can calculate two sensitivities. This first one assuming soil moisture is not affected by warming:
\begin{equation}
\frac{y_{Y}-y_{Z}}{5}
\end{equation}

This next one includes the effect of soil moisture due to warming:
\begin{equation}
\frac{y_{W}-y_{Z}}{5}
\end{equation}

All this said and done, I wonder if it is mathematically equivalent to what you did ... just a different way of showing it/thinking about it. 

\end{document}

\begin{equation}
\end{equation}