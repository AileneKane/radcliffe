\documentclass[11pt,letter]{article}
\usepackage[top=1.00in, bottom=1.0in, left=1.1in, right=1.1in]{geometry}
\renewcommand{\baselinestretch}{1.1}
\usepackage{graphicx}
\usepackage{natbib}
\usepackage{amsmath}

\def\labelitemi{--}
\parindent=0pt

\begin{document}
\bibliographystyle{/Users/Lizzie/Documents/EndnoteRelated/Bibtex/styles/besjournals}
\renewcommand{\refname}{\CHead{}}

\begin{enumerate}
\item Opening paragraphs ...
\begin{enumerate}
\item  \underline{Opening paragraph to review findings}
\begin{enumerate}
\item Soil moisture has not been a focus of previous phenology meta-analyses \cite[e.g.,][]{wolkovich2012}, nor of most multi-species phenology studies in temperate mesic grasslands and forest ecosystem \cite[e.g.,][]{Vitass2021}.
\item But we found that soil moisture strongly affected phenology in temperature systems. 
\item Although multiple environmental conditions affect phenology, interactive effects of soil moisture and temperature were weak for most phenophases. 
\item We might reference back to a point in the intro about what this means -- plants need water to advance and dry soils seem to delay -- moisture as a hidden but potentially limiting factor in temperate systems. 
\end{enumerate}
\item (Maybe then a short paragraph on the forecasting findings?) Soil moisture is and will continue to shift with climate change \citep{berg2017}, so while we found soil moisture had smaller effect than temperature it could have a big impact
\begin{enumerate}
\item Soil moisture is changing -- with some areas, such as the northeastern United States (where many of the experiments were conducted) getting wetter, and other places getting drier \citep{berg2017}
\item Overall our forecasts found temperature will continue to be a dominant controller of phenology, but moisture also matters, especially for certain spp. 
\end{enumerate}
\end{enumerate}
\item {\bf Subheading}: High variation in species-level responses
\begin{enumerate}
\item Despite these overall effects of delays in phenology with soil drying, our results suggest forecasts will need to contend with high variation in species responses, which was not explained by lifeform/functional group.
\begin{enumerate}
\item Our analyses estimate wide variation across species in phenological responses to soil moisture. 
\item We do not find strong differences in soil moisture effects across functional types (Figure S2) but there may be traits associated with the species-level differences in soil moisture effects.  
\end{enumerate}
\item Phenophase differences ... 
\begin{enumerate}
\item Our findings of variation in soil moisture effects across species and phenophase may explain inconsistencies observed in previous studies. For example, \cite{wolkovich2012} found that exotic species advance with precipitation, whereas native species delay at one site (Fargo). 
\item Variance in soil moisture effect varied across phenophases and was lowest for budburst -- perhaps suggesting, across species, species need moisture for budburst? In contrast to temperature where the variation is higher (though the overal effect of temperature is also higher...). 
\item Interactions were weak for budburst and leafout, and stronger for flowering (Fig. \ref{fig:bblofl}).
\end{enumerate}
\end{enumerate}
\item {\bf Subheading}: Forecasting multiple drivers 
\begin{enumerate}
\item Our work here shows that soil moisture affects the phenology of temperate grassland and forest systems (paragraph on putting this in the context of other systems)
\begin{enumerate}
\item highly-cited phenology research in temperate grassland and forest systems has frequently ignored these effects, focusing instead on temperature. Our finding that soil drying has an overall delaying effect on phenology is consistent with \citet{seyed2018}, who found that moisture deficit generally delays phenology in forest ecosystems, and with recent experimental \citep{liu2022} and observational \citep{tao2020} studies in temperate grasslands. 
\item This fits within a larger literature from other systems that have found moisture matters to phenology, including alpine systems dominated by snowpack \citep[e.g.,][]{dunne2004,sherwood2017}, and arid/semi-arid ecosystems where precipitation is known to be more limiting \citep{tao2019}. 
\end{enumerate}
\item So we need forecast both effects for phenology and possibly their impacts on other drivers and limiting resources ... 
\begin{enumerate}
\item Multiple global change factors affect phenology (temperature and soil moisture here, also CO2?, nitrogen, photoperiod)
 \item Soil moisture may actually mediate plant phenology responses to warming and  nitrogen addition, too \citep{liu2022}
 \item Some of observed variable responses to moisture, both here and in previous studies,  may be driven by temporal and spatial variation in the most limiting resource (e.g., temperature vs moisture). (Not sure if we need this?) 
\end{enumerate}
\item To do this, we also need to improve how we relate experiments to `real world'
\begin{itemize}
\item Moving beyond treatments levels to analyze plot-level microclimate- closer to how plants may be experiencing treatments
\item Our study differs from some because we used field-measured soil moisture -- most studies use precipitation \citep[e.g.,][]{tao2022} or gridded moisture products \citep[e.g.,][]{tao2019}. The problems with these proxies are widely known (REF). However, our use of measured soil moisture also created a data limitation, as we were able to use only a subset of all the climate change experiments included in the ExPhen and E3 databases. So, we need more people to measure this!
\item Such new data could help understand how temperature is affected by soil moisture, and how soil moisture is affected by temperature treatments
\end{enumerate}
\end{enumerate}
\end{enumerate}



\end{document}

\item Our work here shows that soil moisture affects the phenology of temperate grassland and forest systems; highly-cited phenology research in these systems has frequently ignored these effects, focusing instead on temperature. Our finding that soil drying has an overall delaying effect on phenology is consistent with \citet{seyed2018}, who found that moisture deficit generally delays phenology in forest ecosystems, and with recent experimental \citep{liu2022} and observational \citep{tao2020} studies in temperate grasslands.  %tao2020:"every 1\% m3/m3 increase in soil moisture led to a green-up date that was 3 to 6 days earlier". 
\item Our study differs from some because we used field-measured soil moisture -- most studies use precipitation \citep[e.g.,][]{tao2022} or gridded moisture products \citep[e.g.,][]{tao2019}. The problems with these proxies are widely known (REF). However, our use of measured soil moisture also created a data limition, as we were able to use only a subset of all the climate change experiments included in the ExPhen and E3 databases. 

\begin{enumerate}
\end{enumerate}



\end{enumerate}
\item {\bf Subheading}: forecasting effects of soil moisture with warming
\begin{enumerate}
\item These findings have important implications for forecasts of phenology
\begin{enumerate}

\end{enumerate}
\item 
\begin{enumerate}

\end{enumerate}
\end{enumerate}
