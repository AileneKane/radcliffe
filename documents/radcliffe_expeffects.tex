\documentclass[11pt]{article}

\usepackage[left=1in,right=1in]{geometry}                
\usepackage{graphicx}
\usepackage{amssymb}
\usepackage{amsbsy}
\usepackage{amsmath}
\usepackage{multirow}
\usepackage{lineno}
\usepackage{caption}
\usepackage{longtable}
\usepackage{setspace}
\usepackage{fancyhdr}
\usepackage{natbib}
\usepackage{subfigure}
\usepackage{booktabs}
\usepackage{lscape}
\usepackage{lscape}
%%%%%%%%%%%%%%%%%%%%%%%%%%%%%%%%%%%%%%%%
\begin{document}

\title{How do warming experiments actually affect microclimate? And what the last generation of experiments could tell us about how to do the next generation (A data synthesis paper}
\author{Ailene (1?), Christy (1?), Lizzie, Yann, Isabelle, Ben, Aaron, Anne Marie, Jeff, Miriam}
\date{June 2016}
\maketitle  %put the fancy title on
%\clearpage
%%%%%%%%%%%%%%%%%%%%%%%
\section{Introduction}
Problem: 
People want to extrapolate climate change experiments (i.e. those that manipulate temperature and precipitation to simulate some future climate scenario) to real life to understand (and forecast) biological impacts of climate change. However, a detailed assessment of exactly how these experiments alter climate, and the extent to which these manipulations accurately model the real world is lacking. 

To address this need, we use plot-level microclimate data from XX climate change experiments. 

\Ex 
\begin{verbatim}
\begin{itemize}
\item Is there seasonal variation in experimental warming effects?
\item How does experimental structure affect microclimate?
\item Do seasonal patterns in experimental climate mirror those in observed climate?
\item (Do treatment effects vary by biome/habitat type?)
\end{itemize}
\end{verbatim}
%%%%%%%%%%%%%%%%%%%%%%%%%
\section{Methods}
Describe all datasets and analyses used.
%%%%%%%%%%%%%%%%%%%%%%%%%
\section{Results}
%%%%%%%%%%%%%%%%%%%%%%%%%
\section{Discussion}
%%%%%%%%%%%%%%%%%%%%%%%%%
\section{Conclusions}
%%%%%%%%%%%%%%%%%%%%%%%%%
 \subsection {}   
    In experiments: plots of sham vs ambient- Ailene (by April 13)
    In experiments: categorical treatments vs reported treatments vs. actual microclimate data- Christy will send Lizzie the actual differences by treatment by April 30; Lizzie will work on models- done by May 13
    In experiments: seasonal variations in climate by reported treatments- Miriam by April 30
    Experiments and observations: figures of climate space: seasonal min and max temperatures for (all? or Gothic and Harvard only for overlapping years) observations and experiments.- Ben/Christy (experimental data from Christy by April 30)
    Plots of treatment effects by biome/habitat type: variance in temperature and precipitation; map onto Whiteaker- Christy by April 30
   
 How climate varies in experiments

    Following on that, try to get projections for a couple sites across a few variables: thinking both about means and variability
    Following on that, how much and how has climate changed in observational data where your experiment is
    But remember that we don't necessarily need all experiments to replicate current or future trends, some test thresholds etc.
    Do analyses with shams
    (Dew)
    Analyses of categorical warming versus spring warming values, versus minT versus maxT

Things everyone should report about their experiments

    Timing of warming treatment applied (e.g., summer application impacts fall events more): exact date it started and how it ran throughout seasons/years
    Day/night variation (across seasons)
    Try to collect climate data at least 2X day, ideally hourly
    Report source populations (in situ, if planted then where collected, germinated where etc.)
    Manipulations of plots
    Shams (yes, no?)

Easy things everyone should try to do with/in experiments

    Measure before and after experiments
    Have shams (effects of perching birds, snowfall etc., intercepted light) and ambient controls, compare them before not having them!

Say somewhere:

    Community standards for climate and phenology (Chuine et al. 2017): How to measure both (climate models forecast air temperature ...)
    Should mention: Monitoring of temperature that is more useful across designs and that is closest to what plants experience

Other

    Useful designs that are not crazy expensive
    Useful in design vs. useful in data collection
    Regression designs for nonlinearity

Phenology-specific

    Try to get data on early stages -- which are more informed by climate drivers
    Be sure to pick best phenological stage for what you are interested in, be expansive -- budset etc.
    What temperatures to measure for phenology (bud temperature, air etc.)
    Phenology as sentinel of how well we manipulate and measure climate in experiments
    Phenology can be observed across scales
    Impacts of different methods on different physiologies (for different species) and how some things may impact chilling versus avoiding frost but not max temp
    Know your system -- have data on what drives phenology over time and space (hopefully first)

%\begin{figure}
%\includegraphics[scale=0.4]{"/figures/"}
%\end{figure}
%%%%%%%%%%%%%%%%%%%%%%%%%%%%%%%%%%%%%%%%
\end{document}
%%%%%%%%%%%%%%%%%%%%%%%%%%%%%%%%%%%%%%%%
