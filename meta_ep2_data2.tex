% Straight up stealing preamble from Eli Holmes 
%%%%%%%%%%%%%%%%%%%%%%%%%%%%%%%%%%%%%%START PREAMBLE THAT IS THE SAME FOR ALL EXAMPLES
\documentclass{article}

%Required: You must have these
\usepackage{Sweave}
\usepackage{graphicx}
\usepackage{tabularx}
\usepackage{hyperref}


%Strongly recommended
  %put your figures in one place
 
%you'll want these for pretty captioning
\usepackage[small]{caption}
\setkeys{Gin}{width=0.8\textwidth}  %make the figs 50 perc textwidth
\setlength{\captionmargin}{30pt}
\setlength{\abovecaptionskip}{0pt}
\setlength{\belowcaptionskip}{10pt}
% manual for caption  http://www.dd.chalmers.se/latex/Docs/PDF/caption.pdf

%Optional: I like to muck with my margins and spacing in ways that LaTeX frowns on
%Here's how to do that
 \topmargin -1.5cm        
 \oddsidemargin -0.04cm   
 \evensidemargin -0.04cm  % same as oddsidemargin but for left-hand pages
 \textwidth 16.59cm
 \textheight 21.94cm 
 %\pagestyle{empty}       % Uncomment if don't want page numbers
 \parskip 7.2pt           % sets spacing between paragraphs
 %\renewcommand{\baselinestretch}{1.5} 	% Uncomment for 1.5 spacing between lines
\parindent 0pt		  % sets leading space for paragraphs
\usepackage{setspace}
%\doublespacing

%Optional: I like fancy headers
\usepackage{fancyhdr}
\pagestyle{fancy}
\fancyhead[LO]{Meta-analysis, episode 2}
\fancyhead[RO]{2016}
 
%%%%%%%%%%%%%%%%%%%%%%%%%%%%%%%%%%%%%%END PREAMBLE THAT IS THE SAME FOR ALL EXAMPLES

%Start of the document
\begin{document}

% \SweaveOpts{concordance=TRUE}
% \bibliographystyle{/Users/Lizzie/Documents/EndnoteRelated/Bibtex/styles/nature.bst}
\title{Data Overview: Predicting Future Springs} % Reconciling Experimental and Observational Approaches for Climate Change Impacts
\author{A. K. Ettinger, E. M. Wolkovich and the Predicting Future Springs Working Group}
%\date{\today}
\maketitle  %put the fancy title on
%\tableofcontents      %add a table of contents
%\clearpage
%%%%%%%%%%%%%%%%%%%%%%%%%%%%%%%%%%%%%%%%%%%%%%%%%%%
%%%%%%%%%%%%%%%% Here we go, boys and girls %%%%%%
\section {Overview of the phenological data}

This is a quick description of the data we will use at our working group. The goal of our working group is to understand (the) underlying cause(s) of the recent finding that results obtained from observational versus experimental studies make radically different predictions for future plant phenology (Wolkovich et al. 2012). The underlying cause of this discrepancy is curently unclear, and to address this we have compiled phenology and climate data for experimental and observational studies. 

There are two main files with the phenological data, one file with the experimental climate data, and a folder with temperature data for the observational sites. They can all be downloaded at \url{https://github.com/AileneKane/radcliffe}. The phenology data files and experimental climate data file are found in the "radmeeting" folder. The temperature data for the observational sites are found in the "Observations/Temp."

\begin{Schunk}
\begin{Sinput}
> setwd("~/GitHub/radcliffe")
> obsdata <- read.csv("radmeeting/obspheno.csv", header=TRUE)
> expdata <- read.csv("radmeeting/exppheno.csv", header=TRUE)
> expclim<-read.csv("radmeeting/expclim.csv", header=TRUE)
\end{Sinput}
\end{Schunk}

We'll walk through the experimental data first. We selected experimental studies that used active warming methods (including above-canopy heating, as well as combined air and soil warming methods) to apply temperature treatments. We additionally limited studies to those that either/both: 1) applied atleast 2 different levels of warming, in addition to controls; and/or 2) measured soil moisture or humidity in all treatments. In many cases those studies that measure soil moisture also manipulate precipitation/moisture through an experimental treatment (i.e. drought and/or increased precipitation treatments).
\subsection{Experimental data}

\begin{Schunk}
\begin{Sinput}
> head(expdata)
\end{Sinput}
\begin{Soutput}
     site plot event year genus species doy genus.species
1 marchin    1   bbd 2011  Acer  rubrum  88   Acer.rubrum
2 marchin    1   bbd 2011  Acer  rubrum  83   Acer.rubrum
3 marchin    1   bbd 2011  Acer  rubrum  96   Acer.rubrum
4 marchin    1   bbd 2011  Acer  rubrum  79   Acer.rubrum
5 marchin    1   bbd 2011  Acer  rubrum  83   Acer.rubrum
6 marchin    1   bbd 2011  Acer  rubrum  80   Acer.rubrum
\end{Soutput}
\end{Schunk}
The phenology data file has the following columns:

site: the first author's name (usually)

plot: the plot or chamber number, given by the author; this can be used to identify the treatment with the "expclim.csv "file, which contains plot and treatment codes, and the "expsiteinfo.csv" file, which contains details on the experimental treatment. For full details on each experiment, see the individual site folders in the "Experiments" folder.

event: phenological event (bbd=first leaf budburst date,lod=first leaf out date, lud= first leaf unfolding date,ffd=first flower date,ffrd=first fruiting date,sd= first seeds dispersing date,col=first date leaf coloration observed, sen=first date senesence observed,drop=leaf drop)

genus and species: 

doy: day of year that the phenological event first occured

Each row is an observation of an individual or plot (whatever the finest scale of observation for that study)

The experimental data come from 9 different sites (see "expsiteinfo.csv" file for details).

\subsection{Observational data}

Next, the observational data. 

\begin{Schunk}
\begin{Sinput}
> head(obsdata)
\end{Sinput}
\begin{Soutput}
    site plot event year doy       date genus   species scrub varetc cult
1 fitter <NA>   ffd 1954 130 1954-05-10  Acer campestre     0     NA   NA
2 fitter <NA>   ffd 1955 131 1955-05-11  Acer campestre     0     NA   NA
3 fitter <NA>   ffd 1956 137 1956-05-16  Acer campestre     0     NA   NA
4 fitter <NA>   ffd 1957 121 1957-05-01  Acer campestre     0     NA   NA
5 fitter <NA>   ffd 1958 128 1958-05-08  Acer campestre     0     NA   NA
6 fitter <NA>   ffd 1959 129 1959-05-09  Acer campestre     0     NA   NA
\end{Soutput}
\end{Schunk}

The observational data come from 15 sites (see XX file for details).

\begin{Schunk}
\begin{Sinput}
> unique(obsdata$site)
\end{Sinput}
\begin{Soutput}
 [1] fitter   harvard  hubbard  konza    niwot    mikesell concord  mohonk   marsham 
[10] fargo    washdc   bolmgren gothic   uwm      rousi   
15 Levels: bolmgren concord fargo fitter gothic harvard hubbard konza ... washdc
\end{Soutput}
\end{Schunk}

\begin{Schunk}
\begin{Sinput}
> table(obsdata$site, obsdata$event)
\end{Sinput}
\begin{Soutput}
              bbd    ffd    fld L75mdoy L95mdoy    lod    lud
  bolmgren      0   1622      0       0       0      0      0
  concord       0   9320      0       0       0      0      0
  fargo         0   4725      0       0       0      0      0
  fitter        0  13721      0       0       0      0      0
  gothic        0 162352      0       0       0      0      0
  harvard     483    284      0       0       0      0      0
  hubbard      72      0      0       0       0     72      0
  konza         0   3403      0       0       0      0      0
  marsham       0   2131    660       0       0      0      0
  mikesell    445      0      0       0       0    549    554
  mohonk        0    673      0       0       0      0      0
  niwot       648    371      0       0       0      0      0
  rousi      1021    147      0       0       0      0      0
  uwm         414      0      0     415     415      0      0
  washdc        0   7455      0       0       0      0      0
\end{Soutput}
\end{Schunk}

\subsection{Species}

\end{document}
